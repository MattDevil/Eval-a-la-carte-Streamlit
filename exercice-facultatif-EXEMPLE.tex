
%%%%%%%%%%%%%%%%%%%%%%%%%%%%%%%%%%%%%%
% Exercice Facultatif
%%%%%%%%%%%%%%%%%%%%%%%%%%%%%%%%%%%%%%

\vspace*{5mm}

\textit{ Attention : cet exercice est facultatif. Il ne faut l'aborder que s'il reste du temps.}

\exo
\begin{enumerate}

\item Calculer $A = \dfrac{10^{37} \times 10 ^{-15}}{10^{50}}$

\item En 1976, le mathématicien Donald KNUTH a inventé une nouvelle opération avec des flèches. Par exemple $10 \uparrow \uparrow 2 = 10 ^{10} $ et $ 2 \uparrow \uparrow 3 = 2^{2^{2}} $. Avec combien de chiffres s'écrit, en écriture décimale, le nombre $10 \uparrow \uparrow 3$ ?  

\end{enumerate}
