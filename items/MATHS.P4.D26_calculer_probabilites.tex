\begin{minipage}{0.99\linewidth}


\exo

\emph{Dans cet exercice, on attend une justification de toutes les réponses.}\par 

Dans une urne il y a 9 boules. 2 sont rouges, 3 sont vertes, et 4 sont bleues.\\

\begin{enumerate}
\item Vanessa choisit au hasard l'une d'entre elles.\\
\begin{enumerate}
\item Quelle est la probabilité que son choix tombe sur l'une des boules bleues ?\\
\item Quelle est la probabilité que son choix tombe sur l'une des boules rouges ou bleues ?\\
\end{enumerate}

\item Cyril choisit une première boule au hasard dans l'urne, il note sa couleur. Puis sans remettre la première boule dans l'urne, il en choisit une deuxième au hasard et note également sa couleur.

\begin{enumerate}
\item Représenter cette expérience aléatoire sous la forme d'un arbre des probabilités, en indiquant sur chaque branche la probabilité associée.

\item Quelle est la probabilité que Cyril obtienne deux boules bleues ?

\item Quelle est la probabilité que Cyril obtienne deux boules de même couleur ?

\end{enumerate}

\end{enumerate}

\end{minipage}

\vspace{0.5cm}
