\begin{minipage}{0.99\linewidth}

\exo

%\titreitem{Utiliser la réciproque du théorème de Thalès.}
\begin{multicols}{2}
\begin{tikzpicture}[line cap=round,line join=round,>=triangle 45,x=1.0cm,y=1.0cm]
\draw [line width=1.2pt] (7.98,0.62)-- (8.,-1.26);
\draw [line width=1.2pt] (8.,-1.26)-- (12.38,-0.74);
\draw [line width=1.2pt] (12.38,-0.74)-- (7.98,0.62);
\draw [line width=1.2pt] (4.5869144358128855,-1.6652065053372829)-- (8.015130105589638,-2.6822299254258932);
\draw [line width=1.2pt] (4.5869144358128855,-1.6652065053372829)-- (8.,-1.26);
\draw [line width=1.2pt] (8.,-1.26)-- (8.015130105589638,-2.6822299254258932);
\draw [color=black] (7.98,0.62)-- ++(-1.5pt,-1.5pt) -- ++(3.0pt,3.0pt) ++(-3.0pt,0) -- ++(3.0pt,-3.0pt);
\draw[color=black] (8.12,0.91) node {$R$};
\draw [color=black] (8.,-1.26)-- ++(-1.5pt,-1.5pt) -- ++(3.0pt,3.0pt) ++(-3.0pt,0) -- ++(3.0pt,-3.0pt);
\draw[color=black] (7.72,-0.77) node {$L$};
\draw [color=black] (12.38,-0.74)-- ++(-1.5pt,-1.5pt) -- ++(3.0pt,3.0pt) ++(-3.0pt,0) -- ++(3.0pt,-3.0pt);
\draw[color=black] (12.68,-0.97) node {$J$};
\draw [color=black] (4.5869144358128855,-1.6652065053372829)-- ++(-1.5pt,-1.5pt) -- ++(3.0pt,3.0pt) ++(-3.0pt,0) -- ++(3.0pt,-3.0pt);
\draw[color=black] (4.12,-1.23) node {$B$};
\draw [color=black] (8.015130105589638,-2.6822299254258932)-- ++(-1.5pt,-1.5pt) -- ++(3.0pt,3.0pt) ++(-3.0pt,0) -- ++(3.0pt,-3.0pt);
\draw[color=black] (7.6,-3.13) node {$Q$};
\end{tikzpicture}

\columnbreak

Sur la figure ci-contre, qui n'est pas en vraie grandeur, on donne : \par
  $LB=2,5$ cm, $LJ=8$ cm, $LR=9,6$ cm et $QR=12,6$ cm.\\
  
  
Démontrer que les droites $(JR)$ et $(BQ)$ sont parallèles.

\end{multicols}

\end{minipage}

\vspace{0.5cm}
