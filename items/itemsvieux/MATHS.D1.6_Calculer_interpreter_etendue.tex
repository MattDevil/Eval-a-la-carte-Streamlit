
\begin{minipage}{0.99\linewidth}

\exo


%\titreitem{Calculer, interpréter l'étendue d'une série statistique.}


\begin{enumerate}

\item Les élèves de 3ème A du collège Coat Mez de Daoulas ont indiqué le nombre de livres qu'ils ont lu pendant le mois de septembre 2017. Voici les résultats de l'enquête : 

\begin{center}

\begin{tabular}{|r|c|c|c|c|c|c|c|}
\hline 
Nombre de livres lus & 0 & 1 & 2 & 3 & 7 & 8 & 15 \\ 
\hline 
Effectif & 12 & 4 & 3 & 3 & 1 & 1 & 1 \\ 
\hline 
\end{tabular} 

\end{center}

\par 

\textbf{Calculer l'étendue de cette série statistique.}

\item La même enquête en 3ème B a donné les résultats suivants : 

\begin{center}

\begin{tabular}{|r|c|c|c|c|c|c|c|}
\hline 
Nombre de livres lus & 4 & 5 & 6 & 7 & 9 & 12 & ... \\ 
\hline 
Effectif & 1 & 4 & 13 & 3 & 1 & 1 & 1 \\ 
\hline
\end{tabular} 
\end{center}

\par

La dernière valeur du tableau a été effacée. Sachant que l'étendue de la série est de 13,\textbf{ quel est le plus grand nombre de livres lus par un élève de 3ème B en septembre 2017 ?}  Justifier votre réponse.


\end{enumerate}


\end{minipage}

\vspace{0.5cm}

