\begin{minipage}{0.99\linewidth}

\exo

% \titreitem{Utiliser le calcul littéral pour prouver, valider ou conjecturer.}

On fait la \textbf{conjecture} \footnote{Propriété mathématiques dont on ne sait pas encore si elle est vraie ou fausse.} suivante : \\

\textit{"Quand on choisit quatre nombres entiers consécutifs \footnote{ Quatre nombres entiers qui se suivent.}, qu'on multiplie le plus petit par le plus grand, puis qu'on ajoute 2, on trouve le même résultat que lorsque qu'on calcule le produit des deux autres nombres."}

\begin{enumerate}

\item Cette conjecture est-elle vraie pour les nombres 4 ; 5 ; 6 et 7 ? Justifier en indiquant le détail des calculs réalisés. 
\item n étant un nombre entier quelconque, n+1 ; n+2 et n+3 sont les nombres entiers qui suivent n.

\begin{enumerate}
\item Développer et réduire l'expression littérale $A = n \times(n+3) +2$.
\item Quelle expression littérale correspond à la phrase \textit{"on calcule le produit des deux autres nombres".}
\item Développer et réduire l'expression littérale obtenue à la question précédente. 
\item En déduire si la conjecture énoncée est vraie.

\end{enumerate} 

\end{enumerate}

\end{minipage}

\vspace{0.5cm}
