\begin{minipage}{0.99\linewidth}

\exo
     
\begin{multicols}{2}

On considère deux cercles de centre $O$ et de diamètres respectifs $120$~cm et $180$~cm.\par

Calculer l'aire de la couronne circulaire comprise entre les deux cercles.
Arrondir le résultat au dixième.

\columnbreak

\begin{center}
   
\begin{tikzpicture}[line cap=round,line join=round,>=triangle 45,x=0.7cm,y=0.7cm]
\clip(9.54,-0.24) rectangle (15.1,5.1);
\draw [line width=1.pt] (12.36,2.44) circle (1.06040181063595cm);
\draw [line width=1.pt] (12.36,2.44) circle (1.677431369684018cm);
\begin{scriptsize}
\draw [color=black] (12.36,2.44)-- ++(-2.5pt,-2.5pt) -- ++(5.0pt,5.0pt) ++(-5.0pt,0) -- ++(5.0pt,-5.0pt);
\draw[color=black] (11.86,2.83) node {$O$};
\end{scriptsize}
\end{tikzpicture}
\end{center}

\end{multicols}

\end{minipage}

\vspace{0.5cm}