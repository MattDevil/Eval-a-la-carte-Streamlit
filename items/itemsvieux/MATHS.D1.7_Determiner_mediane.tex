\begin{minipage}{0.99\linewidth}

\exo


% \titreitem{Déterminer la médiane d'une série statistique.}

\begin{enumerate}

\item Les élèves de 3ème A du collège Coat Mez de Daoulas ont indiqué le nombre de livres qu'ils ont lu pendant le mois de septembre 2017. Voici les résultats de l'enquête :
\begin{center}
 
\begin{tabular}{|r|c|c|c|c|c|c|c|}
\hline 
Nombre de livres lus & 0 & 1 & 2 & 3 & 7 & 8 & 15 \\ 
\hline 
Effectif & 12 & 4 & 3 & 3 & 1 & 1 & 1 \\ 
\hline 
\end{tabular}
 \end{center} 

\textbf{Déterminer le nombre médian de livres lus par les élèves de 3ème A en septembre 2017.} Justifier votre réponse.

\item La même enquête en 3ème B a donné les résultats suivants :

\begin{center}
\begin{tabular}{|r|c|c|c|c|c|c|c|}
\hline 
Nombre de livres lus & 0 & 1 & 2 & 3 & 4 & 6 & 8 \\ 
\hline 
Effectif & 1 & 4 & 13 & 3 & 1 & 1 & 1 \\ 
\hline
\end{tabular} 
\end{center}

\textbf{Déterminer le nombre médian de livres lus par les élèves de 3ème B en septembre 2017.} Justifier votre réponse.



\item Comparer les médianes de ces deux classes et donner votre conclusion à cette enquête.


\end{enumerate}








\end{minipage}

\vspace{0.5cm}
