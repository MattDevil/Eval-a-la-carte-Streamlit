
%%%%%%%%%%%%%%%%%%%%%%%%%%%%%%%%%%%%%%
% Exercices commun à l'ensemble de la classe
%%%%%%%%%%%%%%%%%%%%%%%%%%%%%%%%%%%%%%

\vspace*{5mm}
\textbf{{\large Consignes :}} Calculatrice autorisée pour l'ensemble des exercices. Tout document interdit. Rendre impérativement le sujet avec sa copie double.



\exo
Sur la figure ci-dessous, \textbf{réalisée en vraie grandeur}, toutes les longueurs sont exprimées en centimètre.

\begin{enumerate}

\item Construire, sur la figure ci-dessous, $A_1 B_1 C_1$ l'image du triangle $ABC$ par l'homothétie de centre O et de rapport $0,5$

\item Construire, sur la figure ci-dessous, $A'B'C'$ l'image du triangle $ABC$ par l'homothétie de centre O et de rapport $-1,5$. On indiquera sur la copie les mesures effectuées et les calculs nécessaires à la construction.

\item Prouver, en utilisant les longueurs de ses côtés, que le triangle ABC est rectangle.

\item Que peut-on en déduire pour la nature du triangle $A'B'C'$ ? Justifier.

\item Calculer le périmètre du triangle $ABC$. En déduire, en justifiant, le périmètre de $A'B'C'$

\item Calculer l'aire du triangle $ABC$. En déduire, en justifiant, l'aire du triangle $A'B'C'$.

\end{enumerate}


\begin{tikzpicture}[line cap=round,line join=round,>=triangle 45,x=1.0cm,y=1.0cm]
\clip(3.42,-5.4) rectangle (20.44,3.78);
\draw [line width=1.pt] (6.72,0.46)-- (8.02041782985531,-2.2435002252255143);
\draw [line width=1.pt] (8.02041782985531,-2.2435002252255143)-- (8.74762516891914,1.4353133723914842);
\draw [line width=1.pt] (8.74762516891914,1.4353133723914842)-- (6.72,0.46);
%\draw [line width=4.pt] (16.769373255217033,1.5652503378382718)-- (18.72,-2.49);
%\draw [line width=5.pt] (18.72,-2.49)-- (15.678562246621288,-3.9529700585872263);
%\draw [line width=6.pt] (15.678562246621288,-3.9529700585872263)-- (16.769373255217033,1.5652503378382718);
%\draw [line width=7.pt] (9.12,-0.13)-- (9.770208914927654,-1.4817501126127572);
%\draw [line width=8.pt] (9.12,-0.13)-- (10.13381258445957,0.3576566861957421);
%\draw [line width=9.pt] (10.13381258445957,0.3576566861957421)-- (9.770208914927654,-1.4817501126127572);
\draw[color=black] (6.92,-0.97) node {$3~cm$};
\draw[color=black] (9.2,-0.27) node {$3.75~cm$};
\draw[color=black] (7.58,1.27) node {$2.25~cm$};
%\begin{scriptsize}
\draw [color=black] (6.72,0.46)-- ++(-2.5pt,-2.5pt) -- ++(5.0pt,5.0pt) ++(-5.0pt,0) -- ++(5.0pt,-5.0pt);
\draw[color=black] (6.46,1.07) node {$A$};
\draw [color=black] (8.02041782985531,-2.2435002252255143)-- ++(-2.5pt,-2.5pt) -- ++(5.0pt,5.0pt) ++(-5.0pt,0) -- ++(5.0pt,-5.0pt);
\draw[color=black] (8.14,-2.93) node {$B$};

\draw [color=black] (8.74762516891914,1.4353133723914842)-- ++(-2.5pt,-2.5pt) -- ++(5.0pt,5.0pt) ++(-5.0pt,0) -- ++(5.0pt,-5.0pt);
\draw[color=black] (8.88,1.81) node {$C$};


\draw [color=black] (11.52,-0.72)-- ++(-2.5pt,-2.5pt) -- ++(5.0pt,5.0pt) ++(-5.0pt,0) -- ++(5.0pt,-5.0pt);
\draw[color=black] (11.3,-0.07) node {$O$};
%\draw [color=black] (9.12,-0.13)-- ++(-2.5pt,-2.5pt) -- ++(5.0pt,5.0pt) ++(-5.0pt,0) -- ++(5.0pt,-5.0pt);
%\draw[color=black] (9.32,0.25) node {$A'$};
%\draw [color=black] (10.13381258445957,0.3576566861957421)-- ++(-2.5pt,-2.5pt) -- ++(5.0pt,5.0pt) ++(-5.0pt,0) -- ++(5.0pt,-5.0pt);
%\draw[color=black] (10.34,0.73) node {$C'$};
%\draw [color=black] (9.770208914927654,-1.4817501126127572)-- ++(-2.5pt,-2.5pt) -- ++(5.0pt,5.0pt) ++(-5.0pt,0) -- ++(5.0pt,-5.0pt);
%\draw[color=black] (9.98,-1.11) node {$B'$};
%\draw [color=black] (18.72,-2.49)-- ++(-2.5pt,-2.5pt) -- ++(5.0pt,5.0pt) ++(-5.0pt,0) -- ++(5.0pt,-5.0pt);
%\draw[color=black] (18.97,-2.06) node {$A'_1$};
%\draw [color=black] (16.769373255217033,1.5652503378382718)-- ++(-2.5pt,-2.5pt) -- ++(5.0pt,5.0pt) ++(-5.0pt,0) -- ++(5.0pt,-5.0pt);
%\draw[color=black] (17.01,1.98) node {$B'_1$};
%\draw [color=black] (15.678562246621288,-3.9529700585872263)-- ++(-2.5pt,-2.5pt) -- ++(5.0pt,5.0pt) ++(-5.0pt,0) -- ++(5.0pt,-5.0pt);
%\draw[color=black] (15.93,-3.54) node {$C'_1$};
%\draw [color=black] (7.361437753378709,2.5129700585872263)-- ++(-2.5pt,-2.5pt) -- ++(5.0pt,5.0pt) ++(-5.0pt,0) -- ++(5.0pt,-5.0pt);
%\draw[color=black] (7.61,2.94) node {$C'_2$};
%\draw [color=black] (4.32,1.05)-- ++(-2.5pt,-2.5pt) -- ++(5.0pt,5.0pt) ++(-5.0pt,0) -- ++(5.0pt,-5.0pt);
%\draw[color=black] (4.57,1.48) node {$A'_2$};
%\draw [color=black] (6.270626744782964,-3.005250337838272)-- ++(-2.5pt,-2.5pt) -- ++(5.0pt,5.0pt) ++(-5.0pt,0) -- ++(5.0pt,-5.0pt);
%\draw[color=black] (6.53,-2.58) node {$B'_2$};
%\end{scriptsize}
\end{tikzpicture}

\vspace{5mm}
\hfill {\footnotesize Suite au verso.}
\newpage

\exo

\begin{enumerate}

\item Donner les décompositions en produit de facteurs premiers de $84$ et de $126$.

\item Utiliser la question précédente pour calculer le Plus Grand Commun Diviseur à $84$ et $126$

\item En déduire la fraction irréductible qui est égale à $\dfrac{84}{126}$ 

\end{enumerate}

\exo

On donne les décompositions en produit de facteurs premiers de nombre suivants : $1176 = 2^3 \times 3 \times 7^2$ et $ 756 = 2^2 \times 3^3 \times 7"$.

\begin{enumerate}

\item Quel est le Plus Petit Commun Multiple à $1176$ et $756$ ? Justifier votre réponse. 

\item Utiliser la réponse à la question précédente pour calculer $ \dfrac{247}{1176} + \dfrac{143}{756} $

\end{enumerate}

\exo
\begin{enumerate}

\item Calculer $A = \dfrac{10^{37} \times 10 ^{-15}}{10^{50}}$

\item En 1976, le mathématicien Donald KNUTH a inventé une nouvelle opération avec des flèches. Par exemple $10 \uparrow \uparrow 2 = 10 ^{10} $ et $ 2 \uparrow \uparrow 3 = 2^{2^{2}} $. Avec combien de chiffres s'écrit, en écriture décimale, le nombre $10 \uparrow \uparrow 3$ ?  

\end{enumerate}


\vspace*{5mm}
