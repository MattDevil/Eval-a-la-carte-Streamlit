%Packages à charger

\usepackage[left=1.3cm,right=1.3cm,top=1cm,bottom=1.5cm]{geometry}
\usepackage[utf8]{inputenc}		        % Accents, encodage utf8
\usepackage[T1]{fontenc}		        % Encodage des caractères
\usepackage{lmodern}			        % Choix de la fonte (Latin Modern de D. Knuth)
\usepackage[french]{babel}	        	% Les règles de typo. françaises
\usepackage[autolanguage]{numprint}		% Ecrire les nombres correctement \nombre{12354}
%\usepackage{pythontex}
\usepackage{scratch}
\usepackage{multicol} 					% Multi-colonnes
\usepackage{calc} 						% Calculs 
\usepackage{enumerate}					% Pour modifier les numérotations
\usepackage{enumitem}
\usepackage{graphicx}					% Pour insérer des images
\usepackage{tabularx}					% Pour faire des tableaux
\usepackage{pstricks}
\usepackage{pstricks-add}
\usepackage{pst-eucl, pst-plot, pst-fun} 		% figures géométriques
\usepackage{wrapfig}
\usepackage{pgf,tikz,pgfplots}			% Pour les images et figures géométriques
\pgfplotsset{compat=1.15}
\usetikzlibrary{arrows,calc,fit,patterns,plotmarks,shapes.geometric,shapes.misc,shapes.symbols,shapes.arrows,shapes.callouts, shapes.multipart, shapes.gates.logic.US,shapes.gates.logic.IEC, er, automata,backgrounds,chains,topaths,trees,petri,mindmap,matrix, calendar,folding,fadings,through,positioning,scopes,decorations.fractals,decorations.shapes,decorations.text,decorations.pathmorphing,decorations.pathreplacing,decorations.footprints,decorations.markings,shadows,babel} % Charge toutes les librairies de Tikz
%\usetikzlibrary{arrows}
\usepackage{tkz-tab, tkz-fct, tkz-euclide}	% Géométrie euclidienne avec TikZ
%\usetkzobj{all}
\usepackage{amsmath,amsfonts,amssymb,mathrsfs}  % Spécial math
%\usepackage[squaren]{SIunits}			% Pour les unités (gère le conflits avec  \square de l'extension amssymb)
\usepackage{pifont}						% Pour les symboles "ding"
\usepackage{cancel}						% Pour pouvoir barrer les nombres
\usepackage{url} 			        	% Pour afficher correctement les url
\urlstyle{sf}                          	% qui s'afficheront en police sans serif
\usepackage{eurosym}					% Pour utiliser la commande \euro
\usepackage{fancyhdr,lastpage}         	% En-têtes et pieds
	\pagestyle{fancy}                   % de pages personnalisés
\usepackage{fancybox}					% Pour les encadrés
\usepackage{xlop}						% Pour les calculs posés
%\usepackage{standalone}				% Pour avoir un apercu d'un fichier qui sera utilisé avec un input
\usepackage{multido}					% Pour faire des boucles
\usepackage{hyperref}					% Pour gérer les liens hyper-texte
\usepackage{fourier}
%\usepackage{colortbl} 					% Pour des tableaux en couleur
\usepackage{setspace}					% Pour \begin{spacing}{2.0} \end{spacing}
\usepackage{multirow}					% Pour des cellules multilignes dans un tableau

%Mes réglages
			
\setlength{\parindent}{0mm}								% Pas de retrait en début de paragraphe
\renewcommand{\arraystretch}{1.2}						% Interligne dans les tableaux
\renewcommand{\labelenumi}{\textbf{\theenumi{})}}			% Numérotation en gras
\renewcommand{\labelenumii}{\textbf{\theenumii{})}}		% Numérotation de niveau 2 en gras
\renewcommand{\thesection}{\Roman{section}.}				% Numérotation des sections en chiffres romains
\renewcommand{\thesubsection}{\alph{subsection})}			% Numérotation des sous-sections en lettres
\setlength{\columnsep}{40pt}							% largeur pour séparer les colonnes multicols

%Mes macros

\newcounter{exo}          				% déclaration du numéro d'exo (pas utilisé ici)
\setcounter{exo}{0}   					% initialisation du numero
\newcommand{\exo}{					% \exo
  	\stepcounter{exo}        			% incrémentation du numéro
  	\subsection*{Exercice \no{}\theexo}}
  	
\newcommand{\titreitem}[1]{
\Ovalbox{\makebox[.99\linewidth][l]{{Compétence : {#1} }}}
\vspace{0.3cm}} % Titre des items

